\documentclass[a4paper,english,12pt]{article}
\usepackage{%
	amsfonts,%
	amsmath,%	
	amssymb,%
	amsthm,%
	babel,%
	bbm,%
	biblatex,%
	caption,%
	centernot,%
	color,%
	enumerate,%
	%enumitem,%
	epsfig,%
	epstopdf,%
	etex,%
	framed,%
	fullpage,%
	geometry,%
	graphicx,%
	hyperref,%
	latexsym,%
	mathptmx,%
	mathtools,%
	multicol,%
	pgf,%
	pgfplots,%
	pgfplotstable,%
	pgfpages,%
	proof,%
	psfrag,%
	subfigure,%	
	tikz,%
	times,%
	ulem,%
	url,%
	xcolor%
}	
\definecolor{shadecolor}{gray}{.95}%{rgb}{1,0,0}
\usepackage[mathscr]{eucal}
\usepgflibrary{shapes}
\usetikzlibrary{%
  arrows,%
  backgrounds,%
  chains,%
  decorations.pathmorphing,% /pgf/decoration/random steps | erste Graphik
  decorations.text,%
  matrix,%
  positioning,% wg. " of "
  fit,%
  patterns,%
  petri,%
  plotmarks,%
  scopes,%
  shadows,%
  shapes.misc,% wg. rounded rectangle
  shapes.arrows,%
  shapes.callouts,%
  shapes%
}

\theoremstyle{plain}
\newtheorem*{thm*}{Theorem}
\newtheorem*{exmp*}{}
\newtheorem{thm}{Theorem}[section]
\newtheorem{lem}[thm]{Lemma}
\newtheorem{prop}[thm]{Proposition}
\newtheorem{cor}[thm]{Corollary}

\theoremstyle{definition}
\newtheorem{defn}[thm]{Definition}
\newtheorem{conj}[thm]{Conjecture}
\newtheorem{exmp}[thm]{Example}
\newtheorem{assum}[thm]{Assumptions}
\newtheorem{axiom}[thm]{Axiom}

\theoremstyle{remark}
\newtheorem{rem}[thm]{Remark}
\newtheorem{note}[thm]{Note}
\newtheorem{con}[thm]{Construction}

\newcommand{\eq}[1]{\begin{align*}#1\end{align*}}
\newcommand{\eqn}[1]{\begin{align}#1\end{align}}
\newcommand{\EQ}[1]{\begin{equation*}#1\end{equation*}}
\newcommand{\EQN}[1]{\begin{equation}#1\end{equation}}
\newcommand{\meq}[2]{\begin{xalignat*}{#1}{#2}\end{xalignat*}}
\newcommand{\meqn}[2]{\begin{xalignat}{#1}{#2}\end{xalignat}}
\newcommand{\norm}[1]{\left\lVert#1\right\rVert}
\newcommand{\indep}{\!\perp\!\!\!\perp}
\DeclarePairedDelimiter\abs{\lvert}{\rvert}%
%\DeclarePairedDelimiter\norm{\lVert}{\rVert}%
\newcommand{\tr}{\operatorname{tr}}
\newcommand{\Var}{\operatorname{Var}}
\newcommand{\Cov}{\operatorname{Cov}}

\newcommand{\D}{\mathbb{D}}
\newcommand{\E}{\mathbb{E}}
\newcommand{\N}{\mathbb{N}}
\newcommand{\Q}{\mathbb{Q}}
\renewcommand{\P}{\mathbb{P}}
\newcommand{\R}{\mathbb{R}}
\newcommand{\Z}{\mathbb{Z}}

\newcommand{\C}{\mathcal{C}}
\newcommand{\cG}{\mathcal{G}}

\newcommand{\sB}{\mathscr{B}}
\newcommand{\sE}{\mathscr{E}}
\newcommand{\sF}{\mathscr{F}}
\newcommand{\sL}{\mathscr{L}}
\newcommand{\sS}{\mathscr{S}}
\newcommand{\sT}{\mathscr{T}}
\newcommand{\sX}{\mathscr{X}}
\newcommand{\sY}{\mathscr{Y}}

\renewcommand{\le}{\leqslant}
\renewcommand{\ge}{\geqslant}

% Debug
\newcommand{\todo}[1]{\begin{color}{blue}{{\bf~[TODO:~#1]}}\end{color}}


\makeatletter
\def\th@plain{%
  \thm@notefont{}% same as heading font
  \itshape % body font
}
\def\th@definition{%
  \thm@notefont{}% same as heading font
  \normalfont % body font
}
\makeatother
\date{}
\newcommand\tab[1][1cm]{\hspace*{#1}}
%opening
\title{Lecture-03 : Review of Linear Algebra and Convex Optimization}
\author{}


\begin{document}
\maketitle
\section{Review of Linear Algebra}
\textbf{Reference :}\\
 Appendix A- Foundations of Machine Learning by Mohri,Rostamizadeh and Talwalkar.
\subsection{Definition : Vector Space}
A vector space over $\mathbb{R}$ is a set equipped with following two operations, each satisfying four axioms.\\
I. \textbf{Addition :} $\mathbb{V} \times \mathbb{V} \rightarrow \mathbb{V}$  \\
For any two elements $ u \in \mathbb{V}$ and $v \in \mathbb{V}$ , $\exists$ a resultant element $s (s=u+v) \in \mathbb{V}$\\
\textbf{Satisfying Axioms}:\\
\tab1. Associativity of  addition : $u + (v + w) = (u + v) + w; \tab \forall u,v,w \in \mathbb{V}$\\
\tab2. Commutativity of addition : $u + v = v + u; \tab \forall u,v \in \mathbb{V}$\\
\tab3. Existence of Identity: There exists a zero vector ($0 \in \mathbb{V}$) s.t, $u+0=u; \forall u\in\mathbb{V}$\\
\tab4. Existence of Inverse: For every $u \in \mathbb{V},\exists$ an element $-u\in\mathbb{V}$; s.t, $u+(-u)=0$\\
II. \textbf{Scalar Multiplication}: $\mathbb{R}\times \mathbb{V}\rightarrow \mathbb{V}$\\
For any scalar $\alpha\in\mathbb{R}$ and for any vector $v\in\mathbb{V}$,$\exists$ an element $m (m=\alpha v) \in \mathbb{V}$ \\
\textbf{Satisfying Axioms}:\\ 
\tab1. Compatibility with the field: $a(bu)=(ab)u; \forall a,b \in \mathbb{R}$ \& $u\in\mathbb{V}$\\
\tab2. Existence of Identity : For multiplicative identity element $1 \in \mathbb{R}$, $1u=u; \forall u\in\mathbb{V}$\\
\tab3. Distributivity over vector addition : $\alpha(vu)=\alpha u + \alpha v; \forall \alpha \in \mathbb{R}$ \& $u,v \in\mathbb{V}$ \\
\tab4. Distributivity over field addition : $(\alpha+\beta)u=\alpha u + \beta u; \forall \alpha,\beta \in \mathbb{R}$ \& $u \in\mathbb{V}$ \\
\textbf{Example of Vector Space}:\\
V =  $\mathbb{R}$\\
V =  $\mathbb{R}^N$\\
V =  (c[a,b])\\
\newpage
\subsection{Definition : Inner Product Space}
A Inner Product Space is a vector space equipped with an inner product denoted by $\langle.,.\rangle:V \times V \rightarrow \mathbb{R}$ that satisfying the following axioms\\
\tab 1. Symmetry:\tab $\langle x,y\rangle = \langle y,x\rangle$\\
\tab 2. Linearity:\tab $\langle\alpha x + \beta y, z\rangle = \alpha\langle x,z\rangle +\beta\langle y,z\rangle$\\
\tab 3. Positive Definiteness:\tab $\langle x,x\rangle\geq 0$; $\langle x,x \rangle = 0$ iff $x=0$\\
\textbf{Examples: Inner Product Space}\\
\tab 1.  V =  $\mathbb{R}^N$;  $\langle x,y\rangle$ =$\langle\begin{bmatrix}x_{1} \\. \\ . \\ x_{N}\end{bmatrix}\begin{bmatrix}y_{1} \\. \\ . \\ y_{N}\end{bmatrix}^{T}\rangle$  = $x^{T}y = \displaystyle{\sum_{i}^{N}x_{i}y_{i}}$\\ \\
\tab 2. V =  C($\mathbb{R}^N)$; $\langle f,g\rangle$  = $\displaystyle{\int_{\mathbb{R}^N}(f,g)(t)dt}$\\
\tab 3. Space of Random Variables:   $\langle X,Y\rangle = \mathbb{E}(XY)$\\
\subsection{Definition : Norms}
Norm is a mapping $\norm .:V\rightarrow \mathbb{R_{+}}$ that satisfy the following axioms.\\
\tab 1. Definitiveness: $\norm{X} = 0$ iff  $X=0$\\
\tab 2. Homogenity: $\norm{\alpha X} = \abs{\alpha}\norm{X}$\\
\tab 3. Triangle inequality: $\norm{X+Y}\leq\norm{X}+\norm{Y}$\\
\textbf{Examples of Norms}:\\
\tab 1.  V = $\mathbb{R};\tab\norm{X} = \abs{X}$\\
\tab 2. V =  $\mathbb{R^N}; \tab  \norm{X}_p = \displaystyle{\Big(\sum_{i=1}^{N}\abs {X_{i}}^p\Big)^\frac{1}{p}}$\\
\tab 3. V =  $\mathbb{R^N}; \tab  \norm{X}_2 = \displaystyle{\Big(\sum_{i=1}^{N}\abs {X_{i}}^2\Big)^\frac{1}{2}}$\\
\subsection{Proposition : Holder's Inequality} 
Let p,q $\geq 1$ be conjugate, i.e.  $\frac{1}{p} + \frac{1}{q} =1 $\\
Then, \tab $\abs{\langle x,y\rangle} \leq \norm{x}_p \norm{y}_p$ ; $\forall x,y \in \mathbb{R^N}$\\
\textbf{Proof} :\\
$\log(\frac{1}{p}a^p + \frac{1}{q}b^p) \geq \frac{1}{p}\log(a^p) + \frac{1}{q}\log(b^p)$\tab(Concavity of Log )\\
$\frac{1}{p}a^p + \frac{1}{q}b^p\geq ab$ \tab(Youngs inequality)\\

Let, $a = \frac{\abs {x_{i}}}{\norm x_{p}}$ and $b = \frac{\abs y_{i}}{\norm y_{p}} $\\ 
Hence,  \tab $\frac{1}{p} \frac{\abs x_{i}}{\norm x_{p}^p} + \frac{1}{q} \frac{\abs y_{i}}{\norm y_{p}^q} \geq \frac{\abs x_i\abs y_i}{\norm{x}_p\norm{y}_q}$\\
Summing over $i\in[1,N]$,  we get\\
$1 \geq \displaystyle{\sum_{i=1}^{N} \frac{\abs x_i\abs y_i}{\norm{x}_p\norm{y}_q}}  \geq \frac{\displaystyle{\sum_{i=1}^{N} \abs x_i\abs y_i}}{\norm{x}_p\norm{y}_q} \geq \frac{\abs{\langle x,y \rangle}}{\norm{x}_p\norm{y}_q}$ (Proved.)\\
\textbf{Note:} Let, $f:\mathbb{R^N}\rightarrow\mathbb{R}$\\
$f(y) = f(x) + \displaystyle{\sum_{i=1}^{N} \frac{\partial f}{\partial x_i}(y_i - x_j) + \frac{1}{2}\sum_{i=1}^{N} \frac{\partial^2 f}{\partial x_i \partial x_j}(y_i - x_i)(y_j - x_j)}$\\
$\tab=f(x) + \displaystyle{\langle\begin{bmatrix} \frac{\partial f}{\partial x_1} \\. \\ . \\  \frac{\partial f}{\partial x_N}\end{bmatrix} \begin{bmatrix} {y_1 - x_1} \\. \\ . \\  {y_{N} - x_{N}}\end{bmatrix}\rangle} + \frac{1}{2}(y-x)^T \nabla^2 f(x) (y-x)$\\
\section{Review of Convex Optimization}
\textbf{Reference :}\\
Appendix B- Foundations of Machine Learning by Mohri,Rostamizadeh and Talwalkar.
\subsection{Convex Function} 
A function $f:\mathbb{R^N}\rightarrow\mathbb{R}$ is convex if it's dom$(f)$ is convex and Epi$(f)$ is convex where Epi$(f)$ is defined as follows \\
Epi$(f) = \{(x,y) \in \mathbb{R}^N \times \mathbb{R}: y \geq f(x)\}$\\
For a convex function $f(.)$ ; $f(\alpha x + \bar{\alpha y} ) \leq \alpha f(x) + \bar{\alpha} f(y)$ where $\alpha + \bar{\alpha} =1$ \\
\\
1. If $f$ is differentiable then $f$ is convex iff dom$(f)$ is convex and \\ $f(y) -f(x) \geq \langle\nabla f(x), y-x\rangle; \forall x,y \in$ dom($f$)\\
\textbf{Proof} : $ f(y) - f(x) = \langle\nabla f(x) , y-x\rangle + \frac{1}{2}(y-x)^T \nabla^2 f(x) (y-x) \geq \langle\nabla f(x) , y-x\rangle$\\
2. If $f$ is twice differentiable then $f$ is convex iff dom$(f)$ is convex and it's Hessian is positive semi definite : $\nabla^2 f(x)\succeq 0; \forall x \in$ dom$(f)$\\
\\\textbf{Examples of Convex Function}:\\
1. Linear Function:  $f(x) = \langle w,x\rangle$; where $f:\mathbb{R^N}\rightarrow\mathbb{R}$\\
2. Quadratic Function: $f(x) = x^TAx$; where A is positive semi definite \\
3. Abs Maximum $f(x) = \max \abs X_{i \in N} = \norm{X}_\infty$\\
\subsection{Lemma : Composition of Functions} 
Let, $h(.):\mathbb{R}\rightarrow\mathbb{R};g(.):\mathbb{R}^N\rightarrow\mathbb{R}$ and $f:\mathbb{R}^N\rightarrow\mathbb{R}$ ; $\forall x \in \mathbb{R}^N$ where $f(x)$ is defined by $f(x) = h(g(x))$, then following inequalities are valid\\
1. If h is a convex and non decreasing and g is convex,$\Longrightarrow f(.)$ is convex\\
\textbf{Proof}:\tab As $g(.)$ is convex :  $g(\alpha x + \bar{\alpha}y )\leq \alpha g(x) + \bar{\alpha} g(y)$ \\
\tab \tab Now, $h(g(\alpha x + \bar{\alpha}y ))\leq h(\alpha g(x) + \bar{\alpha} g(y))\leq \alpha h(g(x)) + \bar{\alpha} h(g(y))$\\
Similarly,\\
2. If h is a convex and non increasing and g is concave,$\Longrightarrow f(.)$ is convex\\
3. If h is a concave and non decreasing and g is concave,$\Longrightarrow f(.)$ is concave\\
4. If h is a concave and non increasing and g is convex,$\Longrightarrow f(.)$ is concave\\
\subsection{Jensen's Inequality:} 
Let $X\in C \subset \mathbb{R}^N$ be a r.v with finite mean and $f:C\rightarrow \mathbb{R}$ is convex,\\
Then $\mathbb{E}[X] \in C$,   $\mathbb{E}[f(X)]\leq \infty$ and $f(\mathbb{E}[X]) \leq \mathbb{E}[f(X)]$\\
\textbf{Proof}: \tab {$f(\displaystyle{\sum_{i=1}^{m}\alpha_{i}x_{i}}) \leq \sum_{_i=1}^{m}\alpha_{i}f(x_{i})$ \\where $\alpha_{i}$s could be interpreted as probabilities as $\alpha_{i}\geq0$ and $\displaystyle{\sum_{_i=1}^{m}\alpha_{i}}=1$
	
\subsection{Constrained Optimization}
Let $f:\mathbb{R^N}\rightarrow\mathbb{R}$ and $g_{i}:\mathbb{R^N}\rightarrow\mathbb{R}, i \in [m]$ \\
\textbf{Principle Problem:} $\min f(x)$ s.t. $g_{i}(x)\leq 0;\forall i \in [m]$\\
Note: Let $p^*$ be the optimum value for the above problem.\\
\\ \textbf{Definition : Lagrangian}\\
If $x \in \mathbb{R}^N$ and $\alpha \in \mathbb{R_{+}}^M$, then Lagrangian $\mathscr{L}(x,\alpha):\mathbb{R}^N\times\mathbb{R_{+}}^M\rightarrow\mathbb{R}$ associated with the principal problem is defined as,   $\mathscr{L}(x,\alpha) = f(x) + \displaystyle{\sum_{i=1}^{m}\alpha_i g_i(x)}$;\\
The variables $\alpha \in \mathbb{R_{+}}^M$ are called Lagrange or Dual Variables.
\subsection{Dual Function:}
$F:\mathbb{R^M_+}\rightarrow\mathbb{R}$ defined as $F(\alpha) = \inf\mathscr{L}(x,\alpha)$ where $x\in \mathbb{R^N}$\\
\textbf{Remarks}:\\
1. F is concave in $\alpha$\\
2. $F(\alpha) \leq \mathscr{L}(x,\alpha)\leq f(x)$\\
$\Longrightarrow F(\alpha)  \leq \adjustlimits \inf_{x\in \mathbb{R^N}} f(x) = p^*$\\
$\Longrightarrow F(\alpha) \leq p^*$ such that $g_i(x) \leq 0$\\
\subsection{Dual Problem:}
Max $F(\alpha)$ such that $\alpha \in \mathbb{R_{+}}^M$; \tab Let $d^*$ be the optimal value of this dual problem.\\
Remarks:\\
1.Dual problem is always convex.\\
2.$d^* \leq p^*$\\
3.$(p^* - d^*)$ is called duality gap. When $d^* = p^*$, it is known as strong duality. It holds for convex optimization problems where constraints are qualifying.\\
\subsection{Strong Constraint Qualification:}
Assume that \textbf{int}$(\mathscr{X}) \neq \phi$, then the strong constraint qualification or \textbf{Slater's Condition} is defined as $\exists\bar{x} \in int(\mathscr{X})$, such that $g(\bar x) < 0$\\
\subsection{Weak Constraint Qualification:}
Assume that \textbf{int}$(\mathscr{X}) \neq \phi$, then the strong constraint qualification or \textbf{weak Slater's Condition} is defined as $\exists\bar{x} \in int(\mathscr{X}):\forall i \in [1,m],(g_i(\bar x) < 0)\vee(g_i(\bar x) = 0 \wedge g_i$ affine)\\
\subsection{Saddle Point: Sufficient Condition:}
Let P be a constrained optimum problem over $\mathscr{X} = \mathbb{R^N}$ If $(x^*,\alpha^*)$ is a saddle point of the associated Lagrangian, i.e. $\forall x\in \mathbb{R^N}, \forall\alpha\geq0,$
$\mathscr{L}(x^*,\alpha)) \leq \mathscr{L}(x^*,\alpha^*)\leq \mathscr{L}(x,\alpha^*)$\\
Then, $(x^*,\alpha^*)$ is a saddle point of P.\\
\subsection{Saddle point-Necessary Condition}
\begin{itemize}
	
	\item Assume that $f$ and $g_i$, $i\in [1,m]$ are  convex functions and Slater's condition holds, then if x is a solution of the constrained optimization problem, then there exists $\alpha\geq 0$ s.t $(x,\alpha)$ is a saddle point of the Lagrangian.\\
	\item Assume that $f$ and $g_i$, $i\in [1,m]$ are  convex differentiable functions and weak Slater's condition holds, then if x is a solution of the constrained optimization problem, then there exists $\alpha\geq 0$ s.t $(x,\alpha)$ is a saddle point of the Lagrangian.
	
\end{itemize}


\subsection{Karush-Kuhn-Tucker's Theorem}
Let $f,g_i : \mathscr{X}\rightarrow\mathbb{R},\forall i\in[1,m]$ are convex and differentiable function and that the constrains are qualified. Then $\bar{x}$ is a solution of the constrained problem iff\\ $\exists \bar{\alpha}\geq0$ s.t. \\
1. $\nabla_x \mathscr{L}(\bar{x},\bar{\alpha}) = \nabla_xf(\bar{x}) + \langle\bar{\alpha},\nabla_x g(\bar{x})\rangle =0 $\\
2. $\nabla_\alpha \mathscr{L}(\bar{x},\bar{\alpha}) = g(\bar{x}) \leq 0$\\
3. $\langle\bar{\alpha},g(\bar{x})\rangle =0 $\\


\end{document}