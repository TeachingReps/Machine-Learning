\documentclass[a4paper,english,12pt]{article}
\usepackage{%
	amsfonts,%
	amsmath,%	
	amssymb,%
	amsthm,%
	babel,%
	bbm,%
	biblatex,%
	caption,%
	centernot,%
	color,%
	enumerate,%
	%enumitem,%
	epsfig,%
	epstopdf,%
	etex,%
	framed,%
	fullpage,%
	geometry,%
	graphicx,%
	hyperref,%
	latexsym,%
	mathptmx,%
	mathtools,%
	multicol,%
	pgf,%
	pgfplots,%
	pgfplotstable,%
	pgfpages,%
	proof,%
	psfrag,%
	subfigure,%	
	tikz,%
	times,%
	ulem,%
	url,%
	xcolor%
}	
\definecolor{shadecolor}{gray}{.95}%{rgb}{1,0,0}
\usepackage[mathscr]{eucal}
\usepgflibrary{shapes}
\usetikzlibrary{%
  arrows,%
  backgrounds,%
  chains,%
  decorations.pathmorphing,% /pgf/decoration/random steps | erste Graphik
  decorations.text,%
  matrix,%
  positioning,% wg. " of "
  fit,%
  patterns,%
  petri,%
  plotmarks,%
  scopes,%
  shadows,%
  shapes.misc,% wg. rounded rectangle
  shapes.arrows,%
  shapes.callouts,%
  shapes%
}

\theoremstyle{plain}
\newtheorem*{thm*}{Theorem}
\newtheorem*{exmp*}{}
\newtheorem{thm}{Theorem}[section]
\newtheorem{lem}[thm]{Lemma}
\newtheorem{prop}[thm]{Proposition}
\newtheorem{cor}[thm]{Corollary}

\theoremstyle{definition}
\newtheorem{defn}[thm]{Definition}
\newtheorem{conj}[thm]{Conjecture}
\newtheorem{exmp}[thm]{Example}
\newtheorem{assum}[thm]{Assumptions}
\newtheorem{axiom}[thm]{Axiom}

\theoremstyle{remark}
\newtheorem{rem}[thm]{Remark}
\newtheorem{note}[thm]{Note}
\newtheorem{con}[thm]{Construction}

\newcommand{\eq}[1]{\begin{align*}#1\end{align*}}
\newcommand{\eqn}[1]{\begin{align}#1\end{align}}
\newcommand{\EQ}[1]{\begin{equation*}#1\end{equation*}}
\newcommand{\EQN}[1]{\begin{equation}#1\end{equation}}
\newcommand{\meq}[2]{\begin{xalignat*}{#1}{#2}\end{xalignat*}}
\newcommand{\meqn}[2]{\begin{xalignat}{#1}{#2}\end{xalignat}}
\newcommand{\norm}[1]{\left\lVert#1\right\rVert}
\newcommand{\indep}{\!\perp\!\!\!\perp}
\DeclarePairedDelimiter\abs{\lvert}{\rvert}%
%\DeclarePairedDelimiter\norm{\lVert}{\rVert}%
\newcommand{\tr}{\operatorname{tr}}
\newcommand{\Var}{\operatorname{Var}}
\newcommand{\Cov}{\operatorname{Cov}}

\newcommand{\D}{\mathbb{D}}
\newcommand{\E}{\mathbb{E}}
\newcommand{\N}{\mathbb{N}}
\newcommand{\Q}{\mathbb{Q}}
\renewcommand{\P}{\mathbb{P}}
\newcommand{\R}{\mathbb{R}}
\newcommand{\Z}{\mathbb{Z}}

\newcommand{\C}{\mathcal{C}}
\newcommand{\cG}{\mathcal{G}}

\newcommand{\sB}{\mathscr{B}}
\newcommand{\sE}{\mathscr{E}}
\newcommand{\sF}{\mathscr{F}}
\newcommand{\sL}{\mathscr{L}}
\newcommand{\sS}{\mathscr{S}}
\newcommand{\sT}{\mathscr{T}}
\newcommand{\sX}{\mathscr{X}}
\newcommand{\sY}{\mathscr{Y}}

\renewcommand{\le}{\leqslant}
\renewcommand{\ge}{\geqslant}

% Debug
\newcommand{\todo}[1]{\begin{color}{blue}{{\bf~[TODO:~#1]}}\end{color}}


\makeatletter
\def\th@plain{%
  \thm@notefont{}% same as heading font
  \itshape % body font
}
\def\th@definition{%
  \thm@notefont{}% same as heading font
  \normalfont % body font
}
\makeatother
\date{}

\title{Lecture-06: Kernel Methods}
\date{Aug 21, 2018}
\author{Mrugsen}

\begin{document}
\maketitle
\section{RKHS Theorem Proof:}

	$$\Phi(x) = \sX \to \R$$
	$$x'\to K(x,x')$$
	$$i.e.~\Phi_x(x')=K(x,x')$$
	$\H=\bar{\H}_0$
	\\\\
	\textbf{Properties of the $\inner{\cdot}{\cdot}$ :}
	\begin{itemize}
		\item \textbf{Symmetric:} By the definition of PDS.
		\item By PSD, $\inner{f}{f}~\geq 0$
		\item \textbf{Definiteness:} $\inner{f}{\Phi_x}^2~\leq~\inner{f}{f}\inner{\Phi_x}{\Phi_x}$
		\item \textbf{Bilinear:} $\inner{af+bg}{ch+dk}~=~ac\inner{f}{h}+ad\inner{f,k}+bc\inner{g}{h}+bd\inner{g}{k}$ 
	\end{itemize}
    .\\
    \textbf{Reproducing:}
    $$f\in\H_0~:~f=\sum_{i \in I}a_i\Phi_{x_i}$$
    \textbf{Verify:} $f(x)=\inner{f(\cdot)}{K(x,\cdot)}$
    \\\\
    LHS: $$f(x)=\sum_{i \in I}a_i\Phi_{x_i}(x)=\sum_{i \in I}a_i K(x_i,x)$$
    RHS: $$\inner{\sum_{i \in I}a_i\Phi_{x_i}}{\Phi_x}=\sum_{i \in I}a_iK(x_i,x)$$
    \\
    \textbf{Normalization of PDS Kernels:}
    $$K':\sX \times \sX \to \R~~~~PDS~Kernel$$
    $$K:\sX \times \sX \to \R$$
    \[
      K(x,x')=
              \begin{cases}
               0 \text{ if } K(x,x')=0 \text{ or } K'(x,x)=0 \\
               \frac{K'(x,x')}{\sqrt{K'(x,x)\sqrt{K'(x',x')}}}
              \end{cases}
    \]
    \textbf{Remarks :}
    $$K(x,x)=1~~\forall~x\in\sX~~~\{K(x,x)=0\}$$
    \textbf{H.W.}$$~~K'(x,x')=\exp(\frac{\inner{x}{x'}}{\sigma^2})$$
    $$\text{Then show that, }K(x,x')=\exp(\frac{-\norm{x-x'}^2}{2\sigma^2})$$
    \\
    \textbf{Lemma: Normalized PDS Kernels:}\\
    "Let $K'$ be PDS kernel, then the normalized kernel $K$ is also PDS."\\
    \textbf{Proof:}
    $$\text{Without loss of generality, we can assume }K'(x_i,x_i)~\geq~0~~~~\forall~i\in[m]$$
    $$\sum_{i,j=1}^{m}c_ic_jK(x_i,x_j)=\sum c_ic_j\frac{K'(x_i,x_j)}{\sqrt{K'(x_i,x_i)}\sqrt{K'(x_j,x_j)}}$$
    $$\implies\sum c_ic_jK'(x_i,x_j)~\geq~0~\text{ Since K' is PDS, Here is proved, Lets look some other forms.}$$
    $$\implies\sum c_ic_j\frac{\inner{\Phi_{x_i}}{\Phi_{x_j}}}{\norm{\Phi_{x_i}}\norm{\Phi_{x_j}}}$$
    $$\implies\sum c_ic_j\inner{\frac{\Phi_{x_i}}{\norm{\Phi_{x_i}}}}{\frac{\Phi_{x_j}}{\norm{\Phi_{x_j}}}} \text{ which is, }$$
    $$\norm{\sum_{i=1}^{m}\frac{\Phi_{x_i}}{\norm{\Phi_{x_i}}}}^2~\geq~0$$
    \\
    $\to$ Advantages of working with kernel is that no explicit definition of $\Phi$ is needed.\\\\
    $\to$ \textbf{Advantages of working with explicit $\Phi$ are}:
    \begin{itemize}
    	\item For primal method in various optimization problems.
    	\item To derive an approximation based on $\Phi$
    	\item Theoretical analysis where $\Phi$ is more convenient.
    \end{itemize}
\end{document}