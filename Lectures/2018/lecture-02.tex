\documentclass[a4paper,english,12pt]{article}
\usepackage{%
	amsfonts,%
	amsmath,%	
	amssymb,%
	amsthm,%
	babel,%
	bbm,%
	biblatex,%
	caption,%
	centernot,%
	color,%
	enumerate,%
	%enumitem,%
	epsfig,%
	epstopdf,%
	etex,%
	framed,%
	fullpage,%
	geometry,%
	graphicx,%
	hyperref,%
	latexsym,%
	mathptmx,%
	mathtools,%
	multicol,%
	pgf,%
	pgfplots,%
	pgfplotstable,%
	pgfpages,%
	proof,%
	psfrag,%
	subfigure,%	
	tikz,%
	times,%
	ulem,%
	url,%
	xcolor%
}	
\definecolor{shadecolor}{gray}{.95}%{rgb}{1,0,0}
\usepackage[mathscr]{eucal}
\usepgflibrary{shapes}
\usetikzlibrary{%
  arrows,%
  backgrounds,%
  chains,%
  decorations.pathmorphing,% /pgf/decoration/random steps | erste Graphik
  decorations.text,%
  matrix,%
  positioning,% wg. " of "
  fit,%
  patterns,%
  petri,%
  plotmarks,%
  scopes,%
  shadows,%
  shapes.misc,% wg. rounded rectangle
  shapes.arrows,%
  shapes.callouts,%
  shapes%
}

\theoremstyle{plain}
\newtheorem*{thm*}{Theorem}
\newtheorem*{exmp*}{}
\newtheorem{thm}{Theorem}[section]
\newtheorem{lem}[thm]{Lemma}
\newtheorem{prop}[thm]{Proposition}
\newtheorem{cor}[thm]{Corollary}

\theoremstyle{definition}
\newtheorem{defn}[thm]{Definition}
\newtheorem{conj}[thm]{Conjecture}
\newtheorem{exmp}[thm]{Example}
\newtheorem{assum}[thm]{Assumptions}
\newtheorem{axiom}[thm]{Axiom}

\theoremstyle{remark}
\newtheorem{rem}[thm]{Remark}
\newtheorem{note}[thm]{Note}
\newtheorem{con}[thm]{Construction}

\newcommand{\eq}[1]{\begin{align*}#1\end{align*}}
\newcommand{\eqn}[1]{\begin{align}#1\end{align}}
\newcommand{\EQ}[1]{\begin{equation*}#1\end{equation*}}
\newcommand{\EQN}[1]{\begin{equation}#1\end{equation}}
\newcommand{\meq}[2]{\begin{xalignat*}{#1}{#2}\end{xalignat*}}
\newcommand{\meqn}[2]{\begin{xalignat}{#1}{#2}\end{xalignat}}
\newcommand{\norm}[1]{\left\lVert#1\right\rVert}
\newcommand{\inner}[2]{\left\langle#1, #2\right\rangle}
\newcommand{\indep}{\!\perp\!\!\!\perp}
\DeclarePairedDelimiter\abs{\lvert}{\rvert}%
%\DeclarePairedDelimiter\norm{\lVert}{\rVert}%
\newcommand{\tr}{\operatorname{tr}}
\newcommand{\Var}{\operatorname{Var}}
\newcommand{\Cov}{\operatorname{Cov}}

\newcommand{\one}{\mathbbm{1}}
\newcommand{\D}{\mathbb{D}}
\newcommand{\E}{\mathbb{E}}
\renewcommand{\H}{\mathbb{H}}
\newcommand{\N}{\mathbb{N}}
\newcommand{\Q}{\mathbb{Q}}
\renewcommand{\P}{\mathbb{P}}
\newcommand{\R}{\mathbb{R}}
\newcommand{\Z}{\mathbb{Z}}

\newcommand{\bK}{\mathbf{K}}

\newcommand{\cC}{\mathcal{C}}
\newcommand{\cG}{\mathcal{G}}

\newcommand{\sB}{\mathscr{B}}
\newcommand{\sC}{\mathscr{C}}
\newcommand{\sE}{\mathscr{E}}
\newcommand{\sF}{\mathscr{F}}
\newcommand{\sH}{\mathscr{H}}
\newcommand{\sL}{\mathscr{L}}
\newcommand{\sS}{\mathscr{S}}
\newcommand{\sT}{\mathscr{T}}
\newcommand{\sX}{\mathscr{X}}
\newcommand{\sY}{\mathscr{Y}}

\renewcommand{\le}{\leqslant}
\renewcommand{\ge}{\geqslant}

% Debug
\newcommand{\todo}[1]{\begin{color}{blue}{{\bf~[TODO:~#1]}}\end{color}}


\makeatletter
\def\th@plain{%
  \thm@notefont{}% same as heading font
  \itshape % body font
}
\def\th@definition{%
  \thm@notefont{}% same as heading font
  \normalfont % body font
}
\makeatother
\date{}

%opening
\title{Lecture-02: Linear Algebra Review}
\date{Aug 07, 2018}
\author{Shrey Gupta}


\begin{document}
\maketitle


\section{Basic Definitions}

\begin{defn}
The input space ($\sX$) is the set of all possible examples/instances.
\end{defn}

\begin{defn}
The output space ($\sY$) is the set of all possible labels/targets.
\end{defn}

\begin{defn}
A concept $c:\sX \rightarrow \sY$ is a mapping from input space to ouptut space.
\end{defn}

\begin{defn}
The set of all true concepts is called the concept class ($C$).
\end{defn}

\begin{defn}
The set of all possible candidate concepts is called the hypothesis class ($H$).
\end{defn}

\begin{assum}
All examples in $\sX$ are identically and independently distributed (iid) with an underlying distribution $D$.
\end{assum}

\begin{defn}
When training data $\{ (x_1, y_1), (x_2, y_2), ... , (x_n, y_n) \}$ is given such that $y_i = c(x_i)$ where $c$ is the underlying concept, and the program's job is to guess a hypothesis $h \in H$ so as to minimize the generalization error with respect to $c$, it is called supervised learning.
\end{defn}

\begin{defn}
Given a hypothesis $h\in H$, target concept $c$ and an underlying distribution $D$, the generalization error (risk) is defined as
$$ R(h) = \underset{\sX \sim D} P \left[ h(x) \neq c(x) \right] $$
which can also be expressed as an expectation value
$$ R(h) = \underset{\sX \sim D}{\mathbb{E}} 1_{ \left\{ h(x) \neq c(x) \right\}} $$
\end{defn}

\begin{defn}
The empirical error $(\hat{R}(h))$ is defined as
$$ \hat{R}(h) = \frac{1}{m} \sum_{i=1}^m 1_{ \left\{ h(x_i) \neq c(x_i) \right\} }$$
It can be easily shown that $\mathbb{E} \hat{R}(h) = R(h)$ by using the fact that $\mathbb{E} \sum p_i = \sum \mathbb{E} p_i$
\end{defn}

\section{Support Vector Machines}
\subsection{Linear Classification}

\begin{assum}
Let $N>1$, $\sX = \mathbb{R}^N$ and $\sY = \{-1, 1\}$. Let the target function be $f:\sX \rightarrow \sY$. Let the hypothesis set be defined by
$$ H = \left\{ \textrm{affine functions of }\sX \right\} $$
Let the training samples be denoted by $S = \left\{ (x_1, y_1), (x_2, y_2), ... , (x_m, y_m) \right\}$ where $ y_i = f(x_i)$, $x_i \in \mathbb{R}$, $x_i \overset{iid} \sim D$

\end{assum}

The problem here is to determine $h \in H$ such that $R_D (h)$ is minimized.
$$ R_D (h) = \underset{\sX \sim D} P \left[ h(x) \neq c(x) \right] $$

To do this, let's plot all the points in $\sX$ in an $N$ dimensional space.

\begin{assum}
The training sample $S$ can be separated into two non-empty sets by a hyperplane. In other words, there exists a hyperplane given by
$$ \langle w,x \rangle + b = 0 $$
such that $S=S_1 \cup S_2$ and $S_1 \cap S_2 = \phi$ where
$$S_1 = \left\{ (x,y): \langle w,x \rangle +b > 0  \right\}$$
$$S_2 = \left\{ (x,y): \langle w,x \rangle +b < 0  \right\}$$

\end{assum}

Let $\langle w,x \rangle + b = 0$ be one such plane. The assumption above confirms the existence of at least one pair $(w,b)$. Normalize $(w,b)$ using $\underset{x_i \in S} {\mathrm{min}} \left\{ \langle w,x_i \rangle + b \right\}$. Then, if the closest point is $x_0$
$$ \langle w,x_0 \rangle + b = 1 $$

Let $\rho$ be the minimum distance of any point to the plane.
$$ \rho = \underset{x_i \in S} {\mathrm{min}} \frac{\langle w,x_i \rangle + b}{\| w \|_2} $$

Since $w$ is independent of choice of $x_i$,
$$ \rho = \underset{x_i \in S} {\mathrm{min}} \frac{\langle w,x_i \rangle + b}{\| w \|_2} = \frac{\underset{x_i \in S} {\mathrm{min}} \left\{\langle w,x_i \rangle + b\right\}}{\| w \|_2} = \frac{1}{\|w\|_2} $$

Hence our original problem statement translates to finding $(w,b)$ so as to maximize $\rho$ which in turn boils down to finding $\mathrm{min} \|w\|_2$ given the constraint
$$ y_i \left(\langle w,x_i \rangle + b \right) \geq 1 $$

\section{More definitions}

\begin{defn}[Gradient]

Let $f:X\subset\mathbb{R}^N\rightarrow\mathbb{R}$, then

$$ \nabla f(x) = \left[  \begin{array}{ccc} \frac{\partial f}{\partial x_1} (x) \\ \vdots \\ \frac{\partial f}{\partial x_N} (x)
\end{array}  \right] $$

\end{defn}

\begin{defn}[Hessian]

Let $f:X\subset\mathbb{R}^N\rightarrow\mathbb{R}$, then

$$ \nabla ^2 f(x) = \left[  \frac{\partial ^2 f}{\partial x_i \partial x_j}  \right]_{1 \leq i, j \leq N} $$

\end{defn}

\begin{defn}[Stationary Point]
If $f$ attains a local extremum at $x' = x$ then $\nabla f(x') = 0$. $x'$ is called a stationary point.
\end{defn}

\begin{defn}[Convex Set]
A set $X$ is called convex if $\forall x,y \in X$ and $\alpha \in [0,1]$,
$$ \alpha x + (1-\alpha) y \in X $$
\end{defn}

\begin{defn}[Convex Hull]
A convex hull of a set $A$ is the smallest convex set including $A$.
$$ conv(A) = \left\{ \sum_{x_i \in A} \alpha _i x_i : 0 \leq \alpha _i \leq 1 , \sum \alpha _i = 1 \right\} $$
\end{defn}

\begin{defn}[Convex Function]
Let $\sX \subset \mathbb{R}^N$ be a convex set. Then $f: \sX\rightarrow\mathbb{R}$ is a convex function if
$$ f\left( \alpha x + (1-\alpha)y \right) \leq \alpha f(x) + (1-\alpha) f(y) $$

If f is differentiable then it is convex if
$$ f(y) - f(x) \geq \langle \nabla f(x), {y-x} \rangle $$

If f is twice differentiable then it is convex if
$$ \nabla ^2 f \geq 0 $$
Or in other words, if $\nabla ^2 f$ is a positive semi-definite matrix.
\end{defn}

\end{document}